\documentclass[ % ドキュメントクラス
	uplatex,%upLaTeXを使う
	a5paper,%A5サイズにする
	papersize%紙のサイズがデフォルトと違う場合、PDFにうまく伝える
	]{jsbook}

%% フォント関連
\usepackage[T1]{fontenc} % フォントでT1を使うこと
\usepackage{textcomp} % フォントでTS1を使うこと
\usepackage[utf8]{inputenc} % ファイルがUTF8であること
\usepackage{newpxtext,newpxmath} % ローマンと数式の字体をPalatinoに基づいた新PXフォントで
\usepackage{zi4}%等幅フォントをInconsolataで
\usepackage[multi,deluxe,jis2004]{otf}
\usepackage[prefernoncjk]{pxcjkcat} % なるべく「半角」扱いで。
\cjkcategory{sym18}{cjk} % sym18 (U+25A0 - U+25FF Geometric Shapes) を和文文字あつかい

%% 図表など
% 図の読みこみのために
\usepackage[dvipdfmx, hiresbb]{graphicx, xcolor}
\usepackage{booktabs} % 表の横罫線
\usepackage{lscape}  % 表などを90度回転させる

%% 囲み枠
\usepackage{tcolorbox}
\tcbuselibrary{breakable} % ページをまたいで分割できるように
\tcbuselibrary{skins} % さまざまな skin を準備
\tcbuselibrary{theorems} % 定理環境

% 枠の定義
\newtcolorbox{summary}{
	sharp corners, % 角を丸くしない
	boxrule=0pt, % ボックスの罫線なし
	colback=green!20!white, % ボックスの背景色
	}

\newtcolorbox{note}[1]{
	breakable, % ページをまたいでボックスを分割
	before skip=20pt plus 2pt minus 2pt, % ボックスの前の空き
	after skip=20pt plus 2pt minus 2pt, % ボックスの後の空き
	boxrule=0.4pt, % ボックスの罫線の太さ
	colframe=black!95, % フレームの色
	colback=white!95, % ボックスの背景色
	fonttitle=\gtfamily\bfseries, % タイトルのフォント
	title=#1 % タイトルのテキスト
	}

	% 参考情報を出力する命令を作る
	\newtcbox{mysbox}{ % box を作る命令
		boxrule=0.4pt, % ボックスの罫線の太さ
		colframe=black, % フレームの色
		colback=black!10, % ボックスの背景色
		top=0mm, 
		bottom=0mm, 
		left=0mm, 
		right=0mm, 
		on line, arc=0.5mm}

	\newcommand{\sanko}[1]{
		\begin{itemize}
			\item[\mysbox{\small\gtfamily 参考}] #1
		\end{itemize}
	}

% エピグラフ
\usepackage{epigraph}
\setlength{\epigraphwidth}{.6\textwidth}


%% ソースコードの出力
\usepackage{listings}
\lstset{ % 色々な設定
	language=R, % プログラミング言語名(ここではR言語)
	basicstyle=\ttfamily, % 基本的にタイプライター体にする
	numbers=left, % 左側に行番号を表示
	numberstyle=\small, % 行番号は小さめに
	numbersep=16pt, % 行番号をどれだけ離すか
	% showspaces=true,% スペースを表示したければtrueにする 
	xleftmargin=25pt, % 左側のマージン
	frame=single, %ソースコードを囲むフレーム
	framesep=10pt, %フレームとコードの間隔
	backgroundcolor=\color[gray]{0.95}, % 背景色
	breaklines=true % 長い行は改行する
}
	\renewcommand{\lstlistingname}{ソースコード}


	%% misc
	\usepackage{okumacro} % 圏点などのために
	\usepackage{pxrubrica} % ルビをつける(okumacroのrubyは使わない)


	%% hyperrefの設定
	\usepackage[dvipdfmx, %
	bookmarks=true, %PDFにしおりをつける
	bookmarksnumbered=true, %しおりに節番号などをつける
	colorlinks=false, %リンクには色をつけない
	hyperfootnotes=false, %脚注からのリンクを作らない
	pdfborder={0 0 0}, % リンクの枠なし
	pdfpagelayout=TwoPageRight, %奇数頁が右側になるような見開きモードで開く
	pdfpagemode=UseNone]{hyperref}

% PDFにしたときのしおりの文字化けを防ぐ
\usepackage{pxjahyper}

%% 索引
\usepackage{makeidx}
\makeindex

%\graphicspath{{images/}}

\begin{document}
	
	\begin{titlepage}
		\pagecolor{black}
		\color{white}
		%\renewcommand{\thefootnote}{\fnsymbol{footnote}}
		%\renewcommand{\thefootnote}{\textcolor{red}{\arabic{footnote} \fnsymbol{footnote}}}
		\renewcommand{\thefootnote}{\textcolor{red}{\fnsymbol{footnote}}}

		\begin{center}
			\vspace*{20mm}
			\fontsize{36pt}{0pt} \selectfont
			現代百合学概論
			\vspace*{5mm}
			\fontsize{21pt}{3pt} \selectfont
			{\sl Overview of modern lilics}
			\vspace{30mm}

			{\LARGE 高名 典雅\footnote[2]{\color{white}正弦工社 (https://seigenkousya.github.io/) <seigenkousya@outlook.jp>}}
			\vspace{2mm}\\
			\color{white}
			{\LARGE 現代百合学概論製作委員会\footnote[3]{\color{white}現代百合学概論製作委員会(https://github.com/Seigenkousya/Overview\_of\_modern\_lilics)}}
			\vspace{10mm}\\

			\begin{flushright}
				\color{white}
				{\rm \Large 2020 02/15}\vspace{2mm}\\
				{\rm \Large ver 0.0.1}
			\end{flushright}

			\vspace{10mm}
		\end{center}
	\end{titlepage}
	% 標題終わり

	\pagecolor{white}
	\color{black}
	\frontmatter

	\chapter{序章}

	本稿では、現代(特に、2010年代以降)における百合作品の評論、分類を行い、また作品が社会に与えた影響や文化の変遷を追う現代百合学を掲げ、その礎として本稿をオープンソースで公開し\footnote{https://github.com/Seigenkousya/Overview\_of\_modern\_lilics}広く意見を集めるものとする。\\
	よって本書の内容は常に可変的であり、ソフトウェアのようにバージョンがつけられ完成することなく自由な議論の元編集されつづける。

	\tableofcontents % 目次

	\mainmatter

	\chapter{現代百合学とは何か}

	\epigraph{尊み秀吉}{明智光秀}

	\begin{summary}
		この章では現代百合学における現代的な百合の定義について論ずる
	\end{summary}

	\section{現代百合学の定義}
	本稿の表題は現代百合学である。\\
	では、現代百合学とは、何を対象にした学問なのか。\\
	
	現代百合学は、現代(特に、2010年代以降)の百合作品群の特有の作品の傾向や文化を明らかにし、分類し体系化することでより作品に対する知見を深めるとともによりより作品との出会いを求める学問である。\\
	
	\newpage
	現代百合学で扱う範囲は以下の通りである。

	\begin{itemize}
		\item 現代的な百合作品の特色
		\item 現代的な百合作品の変遷
		\item 現代的な百合作品の分類
		\item 現代的な百合作品の購読者層
		\item 現代的な百合作品における登場人物の関係性
	\end{itemize}

	\subsection{最初の小節の見出し}

	ここには最初の小節の文章が入る。
	ここには最初の小節の文章が入る。
	ここには最初の小節の文章が入る。
	ここには最初の小節の文章が入る。
	ここには最初の小節の文章が入る。

	\subsubsection{最初の小々節の見出し}

	ここには文章が入る。
	ここには文章が入る。
	ここには文章が入る。
	ここには文章が入る。
	ここには文章が入る。


	\paragraph{段落の見出し}

	ここには文章が入る。
	ここには文章が入る。
	ここには文章が入る。
	ここには文章が入る。
	ここには文章が入る。

	\paragraph{とってもとってもとってもとっても長い段落の見出し}

	ここには文章が入る。
	ここには文章が入る。
	ここには文章が入る。
	ここには文章が入る。
	ここには文章が入る。

	\subsection{第二の小節の見出し}

	ここには第二の小節の文章が入る。
	ここには第二の小節の文章が入る。
	ここには第二の小節の文章が入る。
	ここには第二の小節の文章が入る。
	ここには第二の小節の文章が入る。

	\section{第二の節の見出し}

	ここは第二の節の文章が入る。
	ここは第二の節の文章が入る。
	ここは第二の節の文章が入る。
	ここは第二の節の文章が入る。
	ここは第二の節の文章が入る。

	\subsection{第二の小節の見出し}

	ここには第二の小節の文章が入る。
	ここには第二の小節の文章が入る。
	ここには第二の小節の文章が入る。
	ここには第二の小節の文章が入る。
	ここには第二の小節の文章が入る。
	
	\section{百合学}
	では、百合学では何を行うのか。


	\chapter{現代的な百合作品の起こりとその変遷}

	\begin{summary}
		この章では前章で定義した現代的な百合作品の起こりとその変遷について論ずる。
	\end{summary}

	\section{フォント・文字装飾など}
	{\mcfamily\bfseries 太明朝体}・{\gtfamily ゴシック体}・{\gtfamily\bfseries 太ゴシック体}・{\gtfamily\ebseries 極太ゴシック体}・{\mgfamily 丸ゴシック体} 

	\section{囲み枠}

	\begin{note}{囲み記事のタイトル}
		囲み記事の文章がここに入る。
		囲み記事の文章がここに入る。
		囲み記事の文章がここに入る。
		囲み記事の文章がここに入る。
		囲み記事の文章がここに入る。
		囲み記事の文章がここに入る。
		囲み記事の文章がここに入る。
		囲み記事の文章がここに入る。
	\end{note}


	\sanko{短い参考情報を提示するときのために使う。}

	\begin{lstlisting}[caption=簡単なプログラムの例, label=ls:example]
		x <- c(24, 23, 15, 52, 63)
		mean(x) # 平均を計算
	\end{lstlisting}



	\chapter{現代的な百合作品群が人々に与えた影響}

	\begin{summary}
		この章では現代的な百合作品群が人々に与えた影響について論ずる。
	\end{summary}

	\section{購読層の変化}
	現代的な

\end{document}


